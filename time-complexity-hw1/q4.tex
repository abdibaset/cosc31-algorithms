Exponentials are of different orders 


\begin{parts}
    \part Prove that $2^n = \mathcal{O}(3^n)$
    \begin{customsolutionbox}
         $2^n = \mathcal{O}(3^n)$ if $\exists n_0, c > 0, \forall n_0 > n: 2^n \leq c \cdot 3^n$. The inequality
         can be re-written as follows: \[2^n \leq c \cdot  3^n \equiv \left(\frac{2}{3}\right)^n \leq c\]
        As $0 < \left(\frac{2}{3}\right)^n < 1$ as $\forall n$, hence $2^n = \mathcal{O}(3^n)$.
    \end{customsolutionbox}

    \part Prove that $3^n \not= \mathcal{O}(2^n)$
    \begin{customsolutionbox}
        Suppose not, that's $3^n = \mathcal{O}(2^n)$, then we have that $\exists n_0, c > 0, \forall n_0 > n: 3^n \leq c \cdot 2^n$. The inequality
        can be re-written as follows
        \[3^n \leq c \cdot 2^n \equiv \left(\frac{3}{2}\right)^n \leq c\]
        As $n$ grows to infinity, there exists some $n$ such that $\left(\frac{3}{2}\right)^n > c$ hence the contradition.
    \end{customsolutionbox}
\end{parts}