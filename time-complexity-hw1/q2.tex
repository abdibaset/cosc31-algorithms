Prove that $f(n) = \Omega(g(n))$ if and only if $g(n) = \mathcal{O}(f(n))$

\begin{customsolutionbox}
    This means $f(n) = \Omega(g(n)) \iff g(n) = \mathcal{O}(f(n))$ \\
    
    Starting with $f(n) = \Omega(g(n)) \Rightarrow g(n) = \mathcal{O}(f(n))$, we have that $f(n) = \Omega(g(n))$ if
        \begin{equation}
        \exists n_0, c > 0, \forall n > n_0 : f(n) \geq c \cdot g(n)
        \end{equation}
    This can be re-arranged as follows:
        \begin{equation} 
        \exists n_0, c > 0, \forall n > n_0 : \frac{1}{c} \cdot f(n) \geq g(n)
        \end{equation}
    We see that statement(2) follows the definition of $g(n) = \mathcal{O}(f(n))$, hence $f(n) = \Omega(g(n)) \Rightarrow g(n) = \mathcal{O}(f(n))$ \\
    
    To show $g(n) = \mathcal{O}(f(n)) \Rightarrow f(n) = \Omega(g(n))$, we have $g(n) = \mathcal{O}(f(n))$ if
        \begin{equation}
        \exists n_0', c' > 0, \forall n > n_0' : g(n) \leq c' \cdot f(n)
        \end{equation}
    This can be re-arranged as follows:
        \begin{equation}
        \exists n_0', c' > 0, \forall n > n_0' : \frac{1}{c'} \cdot g(n) \leq f(n)
        \end{equation}
    We see that statement(4) follows the definition of $f(n) = \Omega(g(n))$, hence $g(n) = \mathcal{O}(f(n)) \Rightarrow f(n) = \Omega(g(n))$ \\
    Therefore, $f(n) = \Omega(g(n)) \iff g(n) = \mathcal{O}(f(n))$.
\end{customsolutionbox}
    