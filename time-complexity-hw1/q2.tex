Prove that $f(n) = \Omega(g(n))$ if and only if $g(n) = \mathcal{O}(f(n))$

\begin{customsolutionbox}
    By the definition $f(n) = \Omega (g(n))$ if \[\exists n_0, c > 0, \forall n > n_0 : f(n) \geq c \cdot g(n)\] 
    and $g(n) = \mathcal{O}f(n)$ if 
    \[\exists n_0, c' > 0, \forall n > n_0: g(n) \leq c' \cdot f(n)\] the inequality can be written as: 
    \[\exists n_0, c' > 0, \forall n > n_0: \frac{1}{c'} \cdot g(n) \leq f(n)\]
    which matches the form required for the definition of $f(n) = \Omega(g(n))$ with constant \(c = \frac{1}{c'}\).
\end{customsolutionbox}