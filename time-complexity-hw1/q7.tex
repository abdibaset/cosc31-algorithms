Suppose there are n players with IDs $1, 2, . . . , n$. Every possible pair $(i, j)$ of players competed and the
results are recorded in an $n \times n$ matrix $A$. In particular, in the match between $i$ and $j$, if i won against
j, then $A[i, j]$ is set to $1$ and $A[j, i]$ is set to $0$; on the other hand, if the match resulted in a tie, both
$A[i, j]$ and $A[j, i]$ are set to $0$. For each $i$, since $i$ does not compete with self, $A[i, i]$ has no meaning
and its value can be anything. We call a player i a super athlete if i won against every other player.
Here is a simple algorithm to identify a super athlete, if one exists.

\begin{algorithm}
    \SetAlgoLined
    \caption{super-athlete}
    \KwIn{A positive integer \(n\) and an \(n \times n\) matrix \(A\), as described above.}
    \KwOut{The index \(i\) of a super athlete, or \(\perp\) if no super athlete exists.}

    \For{$i \gets 1$ \KwTo $n$}{
        \If{\text{all entries in the ith row of matrix} $A$, leaving out the entry $A[i, i]$, are $1$}{
            \Return $i$
        }
    }
    \Return $\perp$
\end{algorithm}

\begin{parts}
    \part Prove that there is at most one super athlete?
        \begin{customsolutionbox}
            For a player at row $i$, and player any $j$ where $1 \leq j \leq$ and $i \not=j$, if $A[i, j] = 1$, then 
            $A[j, i] = 0$ because there can only be one winner in a match, therefore, there is atmost one supper player.
        \end{customsolutionbox} 
    \part In the algorithm above, Line 4 involves reading all entries of row $i$, with the exception of the ith
    entry of that row. Thus, Line 4 involves reading $n - 1$ entries of the matrix. Since Line $4$ can be
    executed up to $n$ times, the algorithm might read up to $\Theta(n^2)$ matrix entries. Design a cleverer
    algorithm that reads at most $O(n)$ entries of the matrix

    \begin{customsolutionbox}
        \begin{algorithm}[H]
            \SetAlgoLined
            \caption{Super Player}
            \SetKwFunction{FSuperPlayer}{SuperPlayer}
            \SetKwProg{Fn}{Function}{}{}
            \Fn{\FSuperPlayer{$n, A$}}{
                $super \gets 0$ \\
                \For{$i \gets 1$ \KwTo $n$} {
                    \If{$super \neq i$ \textbf{and} $A[super][i] = 0$}{
                        $super \gets i$
                    }
                } 
                \For{$m \gets 1$ \KwTo $super-1$}{
                    \If{$A[super][m] \neq 1$}{
                        \KwRet $\perp$
                    }
                }
                \KwRet $super$
            }
        \end{algorithm}
    \end{customsolutionbox}
    \part Argue why your algorithm is correct. (You should argue both that the output is correct and that
    it reads at most $\mathcal{O}(n)$ entries of the matrix.)
    \begin{customsolutionbox}
        For the optimized we do two inspections, 1) to identify the likely super players, 2) verify if the likey super 
        player is indeed a super player or not. \\
        
        In the first loop for the algorithm in part b), we keep track of some 
        player at row $super$, we then check if $A[super, i] = 0$ for $1 \leq i \leq n$ and $i \not= super$, and 
        update $super$ to $i$ if the condition is satisfied.
        We do so because we know that $super$ has beaten players in the rows $super+1$ to $i-1$ hence they
        can't be super players. If $A[super, i] = 1$ we continue to inspect, until we find next player who has beaten 
        $super$. Therefore, we have $\mathcal{O}(n)$ iterations. \\

        Having identified a potential super player, we can verify if player at row $super$ is a super player in the 
        second loop. If some column $m$, $A[super, m]=0$ then we return $\perp$, otherwise return 
        player at row $super$. Similarly, we have $super-1$ iterations where the worst case is $super=n$ hence 
        we have $\mathcal{O}(n)$ for the second loop. \\

        overall, the program has $2\cdot\mathcal{O}(n)$ time complexity, and we can ignore the constant $2$ so the 
        time complexity is $\mathcal{O}(n)$.

    \end{customsolutionbox}
\end{parts}