\textbf{Bor\v{u}vka's Algorithm} \\
\begin{parts}
    \part An English explanation of Bor\v{u}vka's Algorithm
        \begin{customsolutionbox}
            Given an undirected, connected graph $G = (V, E)$, Bor\v{u}vka's Algorithm finds a minimum spanning tree (MST) by progressively merging components using the lightest outgoing edges. The algorithm proceeds in phases, each consisting of the following steps:
            \begin{enumerate}
                \item For each component (initially, each vertex is its own component), find the lightest edge connecting it to a different component.
                \item Add all such lightest edges to the MST. This merges some components together.
                \item Repeat the process on the new set of components until only one component remains — the MST.
            \end{enumerate}
        \end{customsolutionbox}
        
    \part Pseudocode for Bor\v{u}vka's Algorithm
        \begin{customsolutionbox}
            \begin{algorithm}[H]
                \caption{Bor\v{u}vka's Algorithm}
                \SetKwFunction{FBoruvka}{Boruvka}
                \SetKwProg{Fn}{Function}{:}{}
                \Fn{\FBoruvka{$G = (V, E)$}}{
                    $T \gets \emptyset$  \\
                    $components \gets$ \texttt{MakeSet}$(V)$ \\
                    
                    \While{$\texttt{numComponents}(components) > 1$}{
                        $lightestEdges \gets \emptyset$
                        
                        \ForEach{$C \in components$}{
                            $minEdge \gets \texttt{null}$
                            
                            \ForEach{$e = (u, v) \in E$ where $u \in C$, $v \notin C$}{
                                \If{$minEdge = \texttt{null}$ \textbf{or} $w(e) < w(minEdge)$}{
                                    $minEdge \gets e$
                                }
                            }
                            
                            \If{$minEdge \ne \texttt{null}$}{
                                $lightestEdges \gets lightestEdges \cup \{minEdge\}$
                            }
                        }
                        
                        \ForEach{$e \in lightestEdges$}{
                            \If{$e$ connects two different components}{
                                $T \gets T \cup \{e\}$ \\
                                \texttt{Union}(components, endpoints of $e$)
                            }
                        }
                    }
                    
                    \Return $T$
                }
            \end{algorithm}
        \end{customsolutionbox}

    \part A proof of correctness for Bor\v{u}vka's Algorithm
            \begin{customsolutionbox}
                We prove Bor\v{u}vka’s Algorithm is correct by showing that it always produces a minimum spanning tree (MST) of the input graph $G = (V, E)$. The key invariant maintained throughout the algorithm is:

                \begin{center}
                    \textit{Each edge added to the tree is a light edge crossing some cut of the current components.}
                \end{center}

                \textbf{Cut Property:} In any connected, undirected, weighted graph, the lightest edge crossing any cut is guaranteed to be part of some MST.

                \begin{itemize}
                    \item In each iteration, Bor\v{u}vka's algorithm identifies the lightest edge connecting each component to another.
                    \item Each such edge crosses a cut between that component and the rest of the graph.
                    \item By the cut property, each edge is safe to add to the MST.
                \end{itemize}

                Hence, all edges added are part of some MST. Since edges are only added between distinct components, cycles are never introduced. If we merge two trees of sizes $n_1$ and $n_2$, each with $n_1 - 1$ and $n_2 - 1$ edges respectively, and add one edge between them, the total becomes:

                \[
                    (n_1 - 1) + (n_2 - 1) + 1 = n_1 + n_2 - 1,
                \]

                which is exactly the number of edges in a tree on $n_1 + n_2$ vertices.

                Because the algorithm keeps merging trees without creating cycles and continues until all vertices are connected, the final result is a spanning tree. Since all added edges are safe (by the cut property), this tree is a valid MST.
            \end{customsolutionbox}
    

    \part The time complexity of Bor\v{u}vka's Algorithm, along with an explanation of how that time complexity is derived.
        \begin{customsolutionbox}
            The time complexity of Bor\v{u}vka’s Algorithm is $O(m \log n)$, where $m = |E|$ is the number of edges and $n = |V|$ is the number of vertices.

            \begin{itemize}
                \item \textbf{Finding the lightest edge for each component:} In each phase, we scan all edges to find the lightest outgoing edge from each component. This takes $O(m)$ time.
                
                \item \textbf{Merging components:} Each added edge connects two distinct components, reducing the number of components. Since the number of components at least halves in each phase, there are at most $O(\log n)$ phases.

                \item \textbf{Total work:} Each phase costs $O(m)$ and there are $O(\log n)$ phases, leading to a total time complexity of $O(m \log n)$.
            \end{itemize}
            Thus, the overall time complexity of Bor\v{u}vka's Algorithm is $O(m \log n)$.
        \end{customsolutionbox}


\end{parts}
