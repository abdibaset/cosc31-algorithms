\textbf{Simpler SCC }
Dr. Murkha claims that the algorithm for strongly connected components would be simpler if it used
the original (instead of the transpose) graph in the second depth-first search and scanned the vertices
in order of increasing finishing times. Does this simpler algorithm always produce correct answers?
Justify your answer.
\begin{customsolutionbox}
    No, the simpler algorithm will not always produce correct answers. Consider the graph below:

    \begin{center}
    \begin{tikzpicture}[->, thick, node distance=2cm and 2.5cm, every node/.style={circle, draw=blue, fill=white, minimum size=0.8cm}]
        \node (A) {A};
        \node (B) [below right=of A] {B};
        \node (C) [below left=of A] {C};

        \draw (A) -- (B);
        \draw (B) to[bend left=20] (A);
        \draw (A) -- (C);
    \end{tikzpicture}
    \end{center}

    The edges are $A \to B$, $B \to A$, and $A \to C$. So $\{A, B\}$ is one strongly connected component (SCC), and $\{C\}$ is another. \\


    Suppose we run the first DFS starting at $A$. It visits $B$ and $C$ from $A$. The finish times will be: $B.f = 3$, $C.f = 5$, and $A.f = 6$. 
    Now, the simpler algorithm proposes to run a second DFS on the original graph, using the order of increasing finishing times: $B$, $C$, $A$. 
    Starting from $B$, the DFS explores $A$ then $C$, and returns the SCC $\{A, B, C\}$. This incorrectly groups all three vertices into one SCC: $\{A, B, C\}$ because 
    there is no path from $C$ to $A$ or $B$. \\

    The correct algorithm uses the transpose graph and scans vertices in decreasing finish time, which correctly identifies SCCs: $\{A, B\}$ and $\{C\}$.
\end{customsolutionbox}
