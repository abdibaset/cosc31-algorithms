\textbf{DFS Counterexamples} \\
Give a counterexample to each of the following statements. As you know, whenever you are asked to
give a counterexample, your answer should meet three criteria: (i) you should give a counterexample,
(ii) you should explain why it is a counterexample, and (iii) your counterexample should be as simple
as possible. \\

\definecolor{processblue}{rgb}{0.0, 0.6, 0.8}

\begin{center}
    \begin{tikzpicture}[->, thick, node distance=2.5cm, every node/.style={circle, draw=processblue, fill=white, minimum size=1.2cm}]
        \node (A) {A};
        \node (B) [below left=of A] {B};
        \node (C) [below right=of A] {C};
    
        \draw [bend right] (A) to (B);
        \draw (B) -- (A);
        \draw (A) -- (C);
    \end{tikzpicture}
    \end{center}
    

\begin{parts}
    \part If a directed graph $G$ contains a path from $u$ to $v$ and if $u.d < v.d$ in a depth-first search of $G$,
    then $v$ is a descendent of $u$ in the depth-first forest produced.
    \begin{customsolutionbox}
        Consider the directed graph $G$ with vertices $\{A, B, C\}$ and edges: $A \rightarrow B$, $B \rightarrow A$ and $A \rightarrow C$.

        Start DFS from vertex $A$, then visit $B$ we have $B.d= 2, B.f = 3$. After finishing exploring B, start DFS from $C$ and $C.d = 4$ and $C.f = 5$. 
        So the path $B \rightarrow C$ exists,
        and $B.d (2) < (4) C.d$, but $C$ is \textbf{not} a descendant of $B$ in the DFS forest because the DFF forest produced by DFS is:
        \begin{center}
            \begin{tikzpicture}[->, thick, node distance=2.5cm, every node/.style={circle, draw=processblue, fill=white, minimum size=1.2cm}]
                \node (A) {A};
                \node (B) [below left=of A] {B};
                \node (C) [below right=of A] {C};
            
                \draw (A) -- (B);
                \draw (A) -- (C);
            \end{tikzpicture}
            \end{center}
    \end{customsolutionbox}

    \part If a directed graph G contains a path from u to v, then any depth-first search of G must result in
    $v.d \leq u.f$.
        \begin{customsolutionbox}
            Similarly consider the directed graph $G$ with vertices $\{A, B, C\}$ and edges: $A \rightarrow B$, $B \rightarrow A$ and $A \rightarrow C$ shown above.
            Start DFS from vertex $A$, then visit $B$ we have $B.d= 2, B.f = 3$. After finishing exploring B, start DFS from $C$ and $C.d = 4$ and $C.f = 5$.
            We see that $B.f = 3$ and $C.d = 4$. Although there is a path from $B$ to $C$, we have $C.d (4) > B.f (3)$ hence the contrary of the statement is true.
        \end{customsolutionbox}        
        

\end{parts}