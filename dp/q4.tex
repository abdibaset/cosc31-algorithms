\textbf{Majority} \\
Given an array $A[1 \ldots n]$ $(n \geq 1)$ of numbers, a majority element is a number M such that more than
half the entries of the array $A$ are equal to $M$ . For example, in the array $[2, 3, 2, 2, 4,-1, 2], M = 2$ is
a majority element. Notice that some arrays do not have a majority element, e.g., $[1, 2, 3, 2]$.
In this problem, you will give a linear time iterative algorithm, $Majority(A[1 \ldots n])$ that outputs a
number $m$, which is guaranteed to be the majority element of $A$ if $A$ has a majority element. \\

In this problem, you will give a linear time iterative algorithm, $Majority(A[1 \ldots n])$ that outputs a
number $m$, which is guaranteed to be the majority element of $A$ if A has a majority element.

\begin{parts}
    \part Specify the formal precondition and postcondition for the $Majority(A[1 \ldots n])$ function.
        \begin{customsolutionbox}
            \underline{Preconditions}: $n \geq 1$, and $A[1 \ldots n]$ is an array of integers \\
            \underline{Post condition}: Terminates, and returns a number $m$ if $A$ has a majority element, otherwise 
            returns $\perp$, $A[1 \ldots n]$remains unchanged within its bound 1 through n
        \end{customsolutionbox}
    \part Give a linear time iterative algorithm for $Majority(A[1 \ldots n])$
        \begin{customsolutionbox}
            \begin{algorithm}[H]
                \caption{Majority Element}
                \SetAlgoLined
                \SetKwFunction{FMajorityElement}{MajorityElement}
                \SetKwProg{Fn}{Function}{:}{}
                \Fn{\FMajorityElement{$A[1 \ldots n]$}} {
                    $m \gets 0$ \\
                    $votes \gets 0$ \\
                    \For{$i \gets 1$ \KwTo $n$}{
                        \If{$votes = 0$}{ 
                            $votes \gets votes + 1$ \\
                            $m \gets A[i]$
                        }
                        \ElseIf{$m = A[i]$} {$votes \gets votes + 1$}
                        \Else { $votes \gets votes - 1$}
                    }
                }
                \vspace{1em}

                $mCount \gets 0$ \\
                \For{$i \gets 1$ \KwTo $n$}{
                    \If{$A[i] = m$} { $mCount \gets mCount + 1$}
                    \If{$mCount > \lfloor\frac{n}{2}\rfloor$} { \KwRet $m$}
                }
                \KwRet $\perp$
            \end{algorithm}
        \end{customsolutionbox}

    \part  Prove the correctness of your algorithm using a loop-invariant.
    \begin{customsolutionbox}
        In both loops, we maintain the following invariants. Throughout, $i \in [1, n+1]$ and $A[1 \ldots n]$ is unchanged.
        
        \underline{Loop 1 Invariants (Majority Candidate Identification)}:
        \begin{itemize}
            \item If $votes = 0$, there is no current candidate for majority element in $A[1 \ldots i-1]$.
            
            \item If $votes > 0$, let $x$ be the true majority element (if one exists):
            \begin{itemize}
                \item If $m = x$, then $m$ is the candidate for majority element in $A[1 \ldots i-1]$.
                \item If $m \ne x$, then $m$ becomes the candidate for majority element in $A[1 \ldots i-1]$ because:
        
                \[
                votes > \frac{i-1}{2}
                \]
        
                which means $m$ appears more than half the time in $A[1 \ldots i-1]$. Therefore, no other element, including $x$, can be majority in $A[1 \ldots i-1]$.
            \end{itemize}
        \end{itemize}
        
        \underline{Loop 1 Proof}:
        \begin{itemize}
            \item \textbf{Base case} ($i = 1$): $votes = 0$, so there is no candidate $m$. Invariant holds.
            \item \textbf{Inductive step}: Assume invariant holds for $i$. For $i' = i + 1$, consider:
            \begin{itemize}
                \item \textbf{Case 1}: $votes = 0$: Assign $m \gets A[i']$ and $votes \gets 1$. $m$ becomes new candidate.
                \item \textbf{Case 2}: $votes > 0$ and $A[i'] = m$: Increment $votes$. $m$ remains candidate.
                \item \textbf{Case 3}: If $votes > 0$ and $A[i'] \ne m$, decrement $votes$. If this makes $votes = 0$, the current candidate $m$ is discarded and a new element is selected in the
                next iteration. This satisfies the invariant condition for $votes = 0$: no current candidate is held until the algorithm selects a 
                new candidate when processing the next element.
            \end{itemize}
        \end{itemize}
        
        \vspace{1em}
        
        \underline{Loop 2 Invariant (Verification)}:
        \begin{itemize}
            \item At any point during the second loop, $mCount \leq \lfloor \frac{n}{2} \rfloor$ until $m$ is confirmed as majority.
        \end{itemize}
        
        \underline{Loop 2 Proof}:
        \begin{itemize}
            \item \textbf{Base case} ($i = 1$): Initially, $mCount = 0 \leq \lfloor \frac{n}{2} \rfloor$.
            \item \textbf{Inductive step}: As $i$ increases, $mCount$ increments only when $A[i] = m$. Once $mCount$ exceeds $\lfloor \frac{n}{2} \rfloor$, $m$ is returned. 
            Otherwise, if the loop finishes without this condition, $\perp$ is returned.
        \end{itemize}
        \end{customsolutionbox}
        
\end{parts}