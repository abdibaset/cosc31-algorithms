\textbf{Bigdonald's} \\
Bigdonald's is considering opening a series of restaurants along Deep Valley Highway (DVH). The n
possible locations are along a straight line and their distances from the start of DVH, in miles and in
increasing order, are $m[1], m[2], \ldots, m[n]$. The constraints are as follows:
\begin{itemize}
    \item At each location, Bigdonald’s may open at most one restaurant. The expected profit from opening
    a restaurant at location $i$ is $p[i]$, where $p[i] > 0$ and $i \in \{1, 2, \ldots, n\}$.
    \item Any two restaurants should be at least $k$ miles apart, where $k$ is a positive integer.
\end{itemize}
Design a dynamic programming algorithm of $O(n)$ time complexity and $O(n)$ space complexity that,
given $k, n, m[1 \ldots n]$, and $p[1 \ldots n]$, outputs the maximum expected total profit (subject to the given
constraints) and the locations where to open the restaurants to realize this maximum expected total
profit. Analyze the time and space complexity of your algorithm.

\begin{customsolutionbox}
    \underline{\textbf{Observations}}: Let $mp$ represent the maximum profit. We observe that:
    \begin{itemize}
        \item The maximum profit at index $i$ is either the maximum profit at $i-1$ or the sum of $p[i]$ and the maximum profit at the latest valid index $j$ 
        before $i$ such that $m[i] - m[j] \geq k$ for $1 \leq j < i \leq n$. The valid index $j$ must be the largest index satisfying this condition, with no other index $r$ for
        $j < r < i$ such that $m[i] - m[r] \geq k$.
        \item These valid indices can be precomputed efficiently using a linear two-pointer scan, which allows us to define a function $validIndex$ giving the valid $j$ for each $i$.
    \end{itemize}

    \underline{\textbf{Notations}}: To implement the above idea, we define:
    \begin{itemize}
        \item $validIndex[i]$: the largest index $j$ such that $j < i$ and $m[i] - m[j] \geq k$.
        \item $mp(i)$: the maximum profit achievable considering locations $1$ through $i$.
    \end{itemize}

    \underline{\textbf{Recurrence}}: The maximum profit at index $i$ can be computed as:
    \[
    mp[i] = \max(mp[i-1],\ p[i] + mp[validIndex[i]])
    \]

    \begin{algorithm}[H]
        \caption{Bigdonald's} 
        \underline{\textbf{Preconditions}}: $n \geq 1$, $k \geq 1$, $m[1 \ldots n]$ is an array of possible locations 
        of restaurants from the start of DVH in miles, where $m[i] < m[i+1]$ for $1 \leq i \leq n$, $k$ is a positive 
        integer of the minimum distance between two restaurants, and $p[1 \ldots n]$ is an array of expected profit from opening a restaurant at
        location $i$, where $p[i] > 0$ for $1 \leq i \leq n$. \\
        \underline{\textbf{Postconditions}}: Terminates, and returns the maximum expected total profit and the locations to open restaurants for 
        maximum expected total profit. \\ \\
        \SetAlgoLined
        \SetKwFunction{FMaxProfitAndLocations}{Bigdonald}
        \SetKwProg{Fn}{Function}{:}{}
        \Fn{\FMaxProfitAndLocations{$m[1 \ldots n], p[1 \ldots n], k$}}{
            set validIndex[$i$] to $-1$ for $1 \leq i \leq n$ \\
            set mp[$i$] to $0$ for $s1 \leq i \leq n$ \\
            define a list of $locations \gets [ ]$ \\
            slowPtr $\gets 1$ \\
            fastPtr $\gets 2$ \\
            \vspace{0.05em}

            \While{$fastPtr < n$}{
                \While{$slowPtr < fastPtr$ \textbf{and} m[fastPtr] - m[slowPtr] $\geq k$}{
                    slowPtr $\gets$ slowPtr + 1 
                    
                }
                validIndex[fastPtr] $\gets$ slowPtr - 1 \\
                fastPtr $\gets$ fastPtr + 1 
            }

            \vspace{1em}

            % \tcp*[l]{compute max profit}
            mp$[1] \gets p[1]$ \\
            append $1$ to $locations$ \\
            \For{$i \gets 2$ \KwTo $n$} {
                mp[$i$] = $\max(mp[i-1], p[i] + mp[validIndex[i]])$ \\
                \If{$mp[i] > mp[i-1]$} {
                    append $i$ to $locations$
                }
            }

            \KwRet {$(mp[n], locations)$}
        }
    \end{algorithm}

    \underline{Time complexity}: The total time complexity is $O(2n) = O(n)$. In the first loop, 
    since the slow pointer only moves forward and is not reset, the inner while loop runs in constant time per iteration of the outer loop. 
    The second loop takes exactly $O(n)$ \\
    \underline{Space complexity}: The space complexity is $O(3n) = O(n)$, we are using 3 arrays for which each 
    can have at most $n$ elements.
\end{customsolutionbox}