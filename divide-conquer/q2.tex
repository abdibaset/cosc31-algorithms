Design by divide-and-conquer a $O(\lg m + \lg n)$-time algorithm that, given a positive integer $k$ and two
sorted arrays a and b of size $m$ and $n$, returns the kth smallest element of the union of the two arrays.
Answer the following:

\begin{parts}
    \part Design a divide-and-conquer algorithm by completing the pseudocode for the method below.
    \begin{customsolutionbox}
        \begin{algorithm}[H]
            \caption{SelectFromTwoSortedArrays}
            \SetAlgoLined 
            \SetKwFunction{FSelectFromTwoSortedArrays}{SelectFromTwoSortedArrays}
            \SetKwProg{Fn}{Function}{:}{}
            \Fn{\FSelectFromTwoSortedArrays{$a, b, i, j, p, q, k$}}{
                \If{$k \leq 0$}{
                    \KwRet $\perp$
                }
                \If{$len(A) = 0$}{
                    \KwRet $b[k]$
                }
                \If{$len(B) = 0$}{
                    \KwRet $a[k]$
                }
                $a_{m} \gets \lfloor\frac{i + j}{2}\rfloor$ \\
                $b_{m} \gets \lfloor\frac{p + q}{2}\rfloor$ \\
        
                $a_{\text{left}} \gets a_m - i + 1$ \\
                $b_{\text{left}} \gets b_m - p + 1$ \\
        
                \If {$a_{\text{left}} + b_{\text{left}} < k$}{
                    \If{$a[a_{m}] < b[b_{m}]$} {
                        \KwRet \FSelectFromTwoSortedArrays{$a, b, a_{m} + 1, j, p, q, k - a_{\text{left}}$}
                    } 
                    \Else {
                        \KwRet \FSelectFromTwoSortedArrays{$a, b, i, j, b_{m} + 1, q, k - b_{\text{left}}$}
                    }
                }
                \ElseIf{ $a_{\text{left}} + b_{\text{left}} > k$}{
                    \If{$a[a_{m}] < b[b_{m}]$} {
                        \KwRet \FSelectFromTwoSortedArrays{$a, b, i, j, p, b_{m} - 1, k$}
                    } 
                    \Else {
                        \KwRet \FSelectFromTwoSortedArrays{$a, b, i, a_{m} - 1, p, q, k$}
                    }
                }
                \Else{
                    \KwRet $\min(a[a_{m}], b[b_{m}])$
                }
            }
        \end{algorithm}
        \end{customsolutionbox}
        
        \part Argue the correctness of your algorithm. It is sufficient to state and justify the key observation
        that your algorithm is based on
        \begin{customsolutionbox}
            The key observation is that the $k$th smallest element in the union of two sorted arrays can be found by 
            comparing the middle elements of the current subarrays. At each step, we determine how many elements lie 
            in the left halves of both arrays (up to their respective midpoints). 
            
            If the total number of elements in these left halves is less than $k$, then the $k$th smallest element 
            must lie outside this region — specifically in the right half of one of the arrays. In this case, we discard 
            the left half of the array with the smaller middle element and adjust $k$ accordingly. 
            
            On the other hand, if the total number of elements in the left halves is at least $k$, then the $k$th smallest 
            element must lie within this region. We can safely discard the right half of the array with the larger middle 
            element, without changing $k$.
            
            This divide-and-conquer process eliminates about half of the elements from consideration in each step, 
            leading to a logarithmic number of recursive calls in the sizes of the arrays. The algorithm terminates 
            once one of the arrays is empty or when the base case is met, returning the correct $k$th smallest element.
        \end{customsolutionbox}
            

        \part Let $T(m, n)$ be the time complexity of your algorithm. State the recurrence for $T(m, n)$, where
        $m$ and $n$ are as defined in the pre-condition.
            \begin{customsolutionbox}
                The algorithm either chops off half of the elements from one of the arrays so it is either $T(\frac{m}{2}, n)$ or 
                $T(m, \frac{n}{2})$. The combine time is dominated by the comparison of the two middle elements, which is $O(1)$.
                Hence, the recurrence relation is:
                \begin{align*}
                    T(m, n) &= \max\left(T\left(\frac{m}{2}, n\right), T\left(m, \frac{n}{2}\right)\right) + O(1) 
                \end{align*}
            \end{customsolutionbox}
        
        \part Prove by substitution that $T(m, n) = O(\lg m + \lg n)$.
            \begin{customsolutionbox}
                As shown above, we have the following recurrence relation:
                \begin{align*}
                    T(m, n) &= \max\left(T\left(\frac{m}{2}, n\right), T\left(m, \frac{n}{2}\right)\right) + O(1)
                \end{align*}
                We claims that the $T(m, n) = O(\log m + \log n)$, which can be written as: $T(m, n) \leq c \cdot(\log m + \log n)$ for some 
                constant $c > 0, n_0 > 0, m_0 > 0$, and $\forall n> n_0$ and $\forall m > m_0$.

                \textbf{Inductive hypothesis}: Assume that $T(a, b) \leq c \cdot (\log a + \log b)$ holds $\forall a < m$, and $\forall b < n$.\\
                \textbf{Inductive case}: $T(m, n) \leq c(\log m + \log n)$.

                Suppose we take $T(m/2, n)$ yields the max value, we know that 
                \begin{align*}
                    T\left(\frac{m}{2}, n\right) &\leq T\left(\frac{m}{2}, n\right) + d \\
                    &= c \cdot (\log \frac{m}{2} + \log n) + d \\
                    &= c \cdot (\log m - 1 + \log n) + d \\
                    &= c \cdot (\log m + \log n) - c + d \\
                \end{align*}
                If we pick $c > d$, we see that $T(m, n) \leq c \cdot (\log m + \log n)$. Therefore, for $c > d, n_0 > 0, m_0 > 0$ 
                and $\forall n> n_0$ and $\forall m > m_0$, we have: $T(m, n) = O(\log m + \log n)$

                Otherwise, we can show that:
                \begin{align*}
                    T\left(m, \frac{n}{2}\right) &\leq T\left(m, \frac{n}{2}\right) + d \\
                    &= c \cdot (\log m + \log \frac{n}{2}) + d \\
                    &= c \cdot (\log m + \log n - 1) + d \\
                    &= c \cdot (\log m + \log n) - c + d \\
                \end{align*}
                If we pick $c > d$, we see that $T(m, n) \leq c \cdot (\log m + \log n)$. Therefore, for $c > d, n_0 > 0, m_0 > 0$ 
                and $\forall n> n_0$ and $\forall m > m_0$, we have: $T(m, n) = O(\log m + \log n)$
            \end{customsolutionbox}
\end{parts}