\textbf{Local Minimum 2-D}\\

\begin{parts}
    \part Clearly describe your algorithm. 
        \begin{algorithm}[H]
            \caption{Local Minimum in a 2-D Matrix}
            \SetAlgoLined
            \SetKwFunction{FFindLocalMinimaTwoDMatrix}{FindLocalMinimaTwoDMatrix}
            \SetKwProg{Fn}{Function}{:}{}
            \Fn{\FFindLocalMinimaTwoDMatrix{$G,\ n$}}{
            
                $mid \gets \left\lfloor \frac{n+1}{2} \right\rfloor$ \\
                
                Identify $\text{minVal}$ and its coordinates $(\text{minRow}, \text{minCol})$ among the middle row, middle column, and boundary edges.

                \If{$\text{minVal}$ is smaller than all valid neighbors}{
                    \KwRet $\text{minVal}$
                }

                Identify and cache valid neighbors of $\text{minVal}$ to be considered: \\
                \quad \textbf{If minVal in middle column:} check left and right neighbors (if within bounds) \\
                \quad \textbf{If minVal in middle row:} check top and bottom neighbors (if within bounds) \\
                \quad \textbf{If on boundary:} \\
                \quad\quad If in column 1: check right neighbor \\
                \quad\quad If in column $n$: check left neighbor \\
                \quad\quad If in row 1: check below \\
                \quad\quad If in row $n$: check above \\
            
                \If{$\text{minVal}$ lies on the middle column, and is not a local minimum} {
                    \If{left neighbor $<$ right neighbor}{
                        \If{left neighbor is above middle row}{
                            \KwRet \FFindLocalMinimaTwoDMatrix($G[1\ldots mid-1, 1 \ldots mid-1], mid-1$)
                        }
                        \ElseIf{left neighbor is below middle row}{
                            \KwRet \FFindLocalMinimaTwoDMatrix($G[mid+1 \ldots n, 1 \ldots mid-1], mid-1$)
                        }
                    }
                    \Else {
                        \If{right neighbor is above middle row}{
                            \KwRet \FFindLocalMinimaTwoDMatrix($G[1 \ldots mid-1, mid+1 \ldots n],\ mid-1$)
                        }
                        \ElseIf{right neighbor is below middle row}{
                            \KwRet \FFindLocalMinimaTwoDMatrix($G[mid+1 \ldots n, mid+1 \ldots n],\ mid-1$)
                        }
                    }
                }
            
                \ElseIf{$\text{minVal}$ lies on the middle row, and is not a local minimum} {
                    \If{upper neighbor $<$ lower neighbor}{
                        \If{upper neighbor is left of middle column}{
                            \KwRet \FFindLocalMinimaTwoDMatrix($G[1 \ldots mid-1,\ 1 \ldots mid-1],\ mid-1$)
                        }
                        \ElseIf{upper neighbor is right of middle column}{
                            \KwRet \FFindLocalMinimaTwoDMatrix($G[1 \ldots mid-1,\ mid+1 \ldots n],\ mid-1$)
                        }
                    }
                    \Else {
                        \If{lower neighbor is left of middle column}{
                            \KwRet \FFindLocalMinimaTwoDMatrix($G[mid+1 \ldots n,\ 1 \ldots mid-1],\ mid-1$)
                        }
                        \ElseIf{lower neighbor is right of middle column}{
                            \KwRet \FFindLocalMinimaTwoDMatrix($G[mid+1 \ldots n,\ mid+1 \ldots n],\ mid-1$)
                        }
                    }
                }
            
                \ElseIf{$\text{minVal}$ lies on the boundary, and is not a local minimum} {
                    Identify neighbor with minimum value. \\
                    Recurse into the quadrant that contains that neighbor, \\
                    and compute the correct submatrix size.
                }

            }
            \end{algorithm}    
    \part Rigorously prove the correctness of your algorithm.
       \begin{customsolutionbox}
            Let the predicate $P(n)$ be true if for any 2-D matrix $G[1 \ldots n,\ 1 \ldots n]$ of distinct integers, where $n \geq 1$, and such that
            $G[i, j] = \infty$ for all $i \in \{0, n + 1\}$ or $j \in \{0, n + 1\}$,
            there exists an index pair $(i, j)$ with $1 \leq i, j \leq n$ such that
            $G[i, j] < G[i-1, j],\quad G[i, j] < G[i+1, j],\quad G[i, j] < G[i, j-1],\quad G[i, j] < G[i, j+1]$.
            That is, $G[i,j]$ is strictly less than all four of its neighbors, including those that lie on the boundary of the matrix, 
            which are defined to have value $\infty$. We will prove that $P(n)$ holds for all $n \in \mathbb{N}$ by strong induction. \\
            
            \textbf{Base case:} For $n = 1$, $G[1, 1]$ is bounded by $\infty$ on all sides, so it is trivially a local minimum. This establishes $P(1)$. \\
            
            \textbf{Inductive step:} Fix $k \geq 1$, and assume $P(a)$ holds for all $1 \leq a \leq k$. 
            We want to prove that $P(k + 1)$ holds.
            
            Let $G'[1 \ldots k + 1,\ 1 \ldots k + 1]$ be any matrix of distinct integers, with boundary values 
            $G'[i, j] = \infty$ for all $i \in \{0, k + 2\}$ or $j \in \{0, k + 2\}$. 
            
            The algorithm examines the middle row, middle column, and boundary edges to identify the global 
            minimum among these. Let $\text{minVal}$ denote this minimum value. We consider the following cases:
            
            \begin{enumerate}
                \item If $\text{minVal}$ is smaller than all its four neighbors, then it is a local minimum and is returned.
                
                \item If $\text{minVal}$ lies on the middle column and is not a local minimum, we compare its left and right neighbors. 
                We recurse into the left half if the left neighbor is smaller, and into the right half otherwise.
                
                \item If $\text{minVal}$ lies on the middle row and is not a local minimum, we compare its top and bottom neighbors. 
                We recurse into the upper half if the top neighbor is smaller, and into the lower half otherwise.
                
                \item If $\text{minVal}$ lies on the boundary and is not a local minimum, we compare its in-bounds neighbors and recurse 
                into the quadrant that contains the neighbor with the smallest value.
            \end{enumerate}
            
            In all cases, the submatrix we recurse into has size at most $\lfloor \frac{k + 1}{2}\rfloor by  \lfloor \frac{k + 1}{2}\rfloor$
            where $\lfloor \frac{k + 1}{2}\rfloor\rfloor \leq k$. 
            By the inductive hypothesis, the recursive call returns a local minimum within that submatrix. This local minimum is 
            also valid in the original matrix because we recurse toward the smallest neighbor of $\text{minVal}$, 
            and any neighbor outside the submatrix is either $\infty$ or was already compared when identifying the global minimum 
            and found to be larger than the minVal, and consequently any value smaller than the minVal. 
            Thus, the returned element is smaller than all four of its neighbors in $G'$. Therefore, $P(k + 1)$ holds. 
            By strong induction, $P(n)$ holds for all $n \in \mathbb{N}$.

            
       \end{customsolutionbox}
    \part State the algorithm\'s time complexity as a recurrence, and prove that the algorithm\'s time complexity is $O(n)$.
       \begin{customsolutionbox}
        The algorithm has a time complexity of $O(n)$. The combine time, $f(n)$, consideratios are $O(1)$ for the comparison of the potenial 
        local minimum with its neighbors, and $O(n)$ for finding the minimum in the middle column for a given recursive call, hence $f(n) = O(n)$.

        For each recursive call, we are decrementing the number of columns and rows of the matrix by half each, hence the recurrence relation, $T(n)$, is as 
        follows:
        \[T(n) = T(\frac{n}{2}) + O(n)\]

        By masters theorem, we have:
        \begin{align*}
            f(n) &= O(n) \\
            n^{\log_b a} &= n^{\log_2 2^0} = 1 \\ 
        \end{align*}
        We see that this fits into case 3, for $\epsilon = 1$:
        \begin{align*}
            f(n) &= \Omega({n^{\log_b a + 1}}) \\
            &= \Omega({n^{\log_2 2^0 + 1}}) \\
            &= \Omega({n}) \\
        \end{align*}
        Checking the regularity condition:
        \begin{align*}
            T(n) &\leq T(\frac{n}{2}) + cn \\
            af\left(\frac{n}{b}\right) &\leq cf(n) \\
            1 \cdot \frac{n}{2} &\leq cn
        \end{align*}
        For some $c$ where $\frac{1}{2} < c < 1$, the regularity condition is satisfied. Hence $T(n)$ is:
        \begin{align*}
            T(n) &= \Theta(f(n)) \\
            &= \Theta(n) \\
        \end{align*}
        Since $T(n)$ is $\Theta(n)$, we can conclude that the algorithm runs in $O(n)$ time.
       \end{customsolutionbox}

\end{parts}