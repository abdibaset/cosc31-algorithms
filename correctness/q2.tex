\textbf{Missing Number} \\
Suppose that $a[1 \ldots n]$ contains all but one of the numbers in $\{1, 2,\ldots, n}$ and $\perp$, in some arbitrary
order. Thus, exactly one number from $\{1, 2, \ldots, n\}$ is missing in $a[1 \ldots n]$ . For example, suppose that
$n = 5$ and $a[1 \ldots 5]$ contains $(4, 5, 1, \perp, 3)$; then, the missing number is 2.

\begin{customsolutionbox}
    \begin{algorithm}[H]
        \caption {MissingNumber }
        \SetAlgoLined 
        \underline{Precondition}: $n\geq 1$; $a[1\ldots n]$ is a permuation of $\{\perp, 1, 2, \ldots, n\} - m$, for some
        $m \in \{1, 2, \ldots, n\}$ \\
        \underline{Postcondition}: Terminates, and returns $m$, and the array $a$ is unchanged outside of the range $1 \ldots n$ \\
        \SetKwFunction{FMissingNumber}{MissingNumber}
        \SetKwProg{Fn}{Function}{:}{}
        \Fn{\FMissingNumber{$a, n$}}{
            expectedSum $\gets \frac{n(n+1)}{2}$ \\ 
            actualSum $\gets 0$ \\

            \For{$i \gets 1$ \KwTo $n$}{
                \If{$a[i] \neq \perp$}{
                    actualSum $\gets$ actualSum + $a[i]$ 
                }
            }
            \KwRet expectedSum - actualSum 
        }
    \end{algorithm}

    \underline{Correctness}: The observation is that, there can only be one missing number from the array $a[1 \ldots n]$, hence
    the difference between the sum of all numbers from $1 \ldots n$ and the sum of all numbers in the array $a$ expect for 
    $\perp$ is the missing number. The algorithm runs in $O(2n) = O(n)$ to compute the two sums, and returns the difference. 
\end{customsolutionbox}