\textbf{Party Planning} \\
Professor Stewart is consulting for the president of a corporation that is planning a company party.
The company has a hierarchical structure; that is, the supervisor relation forms a tree rooted at the
president. The personnel office has ranked each employee with a conviviality rating, which is a real
number. In order to make the party fun for all attendees, the president does not want both an employee
and his or her immediate supervisor to attend. Professor Stewart is given the tree that describes the
structure of the corporation, using the left-child, right-sibling representation (read my notes on the next
page to understand/recall this representation and how to traverse a tree represented in this manner).
Each node of the tree holds, in addition to the pointers, the name of an employee and that employee’s
conviviality ranking (in your algorithm, if you need to introduce a constant number of more fields
in a node, you can assume that they have always been there for you to use them). Give a dynamic
programming algorithm that prints out a guest list that maximizes the sum of the conviviality ratings
of the guests.
Your solution must include all of the steps of dynamic programming emphasized in the lecture.

\begin{customsolutionbox}
    \underline{Clever observation}: For any parent node, we either include it in the guest list and exclude all its children, 
    or exclude it and choose the best possible combination of its children. We maintain two lists: one if the president is included, 
    and one if the president is excluded. We recursively compute the best option for each node.

    \underline{Notation}: Let $x$ be a node in the tree. Define:
    \begin{itemize}
        \item $Include(x)$: Maximum conviviality of subtree rooted at $x$ if $x$ is invited.
        \item $Exclude(x)$: Maximum conviviality of subtree rooted at $x$ if $x$ is not invited.
    \end{itemize}

    \underline{Recurrence}:
    \[
    Include(x) = x.\texttt{val} + \sum_{\text{child } y \text{ of } x} Exclude(y)
    \]
    \[
    Exclude(x) = \sum_{\text{child } y \text{ of } x} \max(Include(y), Exclude(y))
    \]

    \[
    \text{MaxConviviality}(x) =
    \begin{cases}
        0 & \text{if } x=NULL \\
        x.\texttt{val} + \sum\limits_{\text{child } y} \texttt{Exclude}(y), & \text{if } x \text{ is invited} \\
        \sum\limits_{\text{child } y} \max(\texttt{Include}(y), \texttt{Exclude}(y)), & \text{if } x \text{ is not invited}
    \end{cases}
    \]

    \[\text{MaxConviviality}(x) = max(Include(x), Exclude(x))\]
    \[\text{guest list} = \begin{cases}
        include\_root\_list & \text{ if } Include(x) > Exclude(x) \\
        exclude\_root\_list & \text{ otherwise}
    \end{cases}\]

    \begin{algorithm}[H]
        \caption{Compute Max Conviviality and Guest List}
        \SetKwFunction{FFindGuests}{FindGuests}
        \SetKwProg{Fn}{Function}{:}{}
        \Fn{\FFindGuests($root$)}{
            $(inc, inc\_list, exc, exc\_list) \gets$ ComputeConviviality(root) \\
    
            \If{$inc > exc$}{
                print the $inc\_list$
            }
            \Else{
                print the $exc\_list$
            }
        }

        \SetKwFunction{FCompute}{ComputeConviviality}
        \SetKwProg{Fn}{Function}{:}{}
        \Fn{\FCompute{$x$: node}}{
            \If{$x = \texttt{NULL}$}{
                \KwRet $(0,\ [\ ], 0, [\ ])$ \tcp*{(max\_value, guest\_list)}
            }
            \vspace{1em}
            $include\_parent \gets x.\texttt{val}$ \\
            $include\_list \gets [x.\texttt{id}]$ \\
            $exclude\_parent \gets 0$ \\
            $exclude\_list \gets [\ ]$

            \vspace{1em}
            $child \gets x.\texttt{lmc}$ \\
            \While{$child \neq \texttt{NULL}$}{
                $(child\_inc,\ child\_inc\_list,\ child\_exc,\ child\_exc\_list) \gets$ ComputeConviviality($child$) 
                
                $include\_parent \hs += child\_exc$ \\
                $include\_list \hs += child\_exc\_list$ 

                \If{$child\_inc > child\_exc$}{
                    $exclude\_parent \hs += child\_inc$ \\
                    $exclude\_list \hs += child\_inc\_list$
                }
                \Else{
                    $exclude\_parent \hs += child\_exc$ \\
                    $exclude\_list \hs += child\_exc\_list$
                }

                $child \gets child.\texttt{rs}$
            }

        
            \KwRet $(include\_parent, include\_list, exclude\_parent, exclude\_list)$
        }
    \end{algorithm}
        
          
        
    \underline{Time complexity}: $O(n)$ where $n$ is the number of nodes in the tree. \\
    \underline{Space complexity}: $O(n)$ for recursion stack space.
\end{customsolutionbox}

